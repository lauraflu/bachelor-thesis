\chapter{Kerneluri ConnexArray}

\section{Kernel pentru înmulțirea a doi vectori}
\label{sec:kernel-mult-arr}
\lstset{
    basicstyle=\fontsize{9}{9}\selectfont\ttfamily\color{black},
    language=C++,
    directivestyle={\color{black}},
    emph={int,char,double,float,unsigned},
}
\begin{lstlisting}[
    caption={Kernel de inițializare a conținutului registrelor},
    label={lst:init-kernel-mul}
]
void init_kernel(void) {
  BEGIN_KERNEL("init_kernel");
    EXECUTE_IN_ALL(
      R25 = 0;   // locatia din memoria locala de unde se incarca primul vector
      R26 = 511; // locatia din memoria locala de unde se incarca al 2-lea vector
      R30 = 1;   // numarul 1; util in calcule
      R31 = 0;   // numarul 0; util in calcule
    )
}

void multiply_kernel(void) {
  BEGIN_KERNEL("multiply_arr_mat");
    EXECUTE_IN_ALL(
      R1 = LS[R25];         // Pasul 1 - partea 1
      R2 = LS[R26];         // Pasul 1 - partea a 2-a
      R29 = INDEX;          // Incarcarea indexului unui PE
      
      R3 = R1 * R2;         // Pasul 2
      R3 = MULT_HIGH();

      CELL_SHL(R2, R30);    // Pasul 5
      NOP;
      R4 = SHIFT_REG;
      R4 = R1 * R4;         // Pasul 7
      R4 = MULT_HIGH();

      CELL_SHR(R2, R30);    // Pasul 6
      NOP;
      R5 = SHIFT_REG;
      R5 = R1 * R5;         // Pasul 8
      R5 = MULT_HIGH();

      R9 = INDEX;           // Pasul 9 - selectia PE cu index impar
      R9 = R9 & R30;
      R7 = (R9 == R30);
    )

    EXECUTE_WHERE_EQ(       // Instructiunile se executa doar in PE impar
      R3 = R31 - R3;        // Pasul 3

      R4 = R5;              // Pasul 9 - mutarea R5 -> R4
    )

    EXECUTE_IN_ALL(
      REDUCE(R3);           // Pasul 3
      REDUCE(R4);           // Pasul 10
    )
  END_KERNEL("multiply_arr_mat");
}
\end{lstlisting}


\section{Kernel pentru înmulțirea mai multor vectori cu aceeași matrice}
\label{sec:kernel-mult-arr-mat}
\lstset{
    language=C++,
    directivestyle={\color{black}},
    emph={int,char,double,float,unsigned},
}
\begin{lstlisting}[
    caption={Kernel pentru înmulțirea unui vector linie cu o matrice pătratică,},
    label={lst:vector-matrix-kernel}
]
void MUSIC_lin_array_cnx_impl::init_kernel(int size_of_block)
{
  BEGIN_KERNEL("initKernel");
    EXECUTE_IN_ALL(
      R10 = 0;
      R11 = 1;
      R12 = 0;
      R13 = 900;
      R14 = size_of_block;      // Equal to ARR_SIZE_C; dimension of the blocks
                                // on which reduction is performed at once
      R8 = INDEX;               // Select only the odd PEs
    )
  END_KERNEL("initKernel");
}

void MUSIC_lin_array_cnx_impl::init_index(void)
{
  BEGIN_KERNEL("initIndex");
    EXECUTE_IN_ALL(
      R2 = LS[R13];             // load input matrix
      REDUCE(R0);
    )
  END_KERNEL("initIndex");
}

void MUSIC_lin_array_cnx_impl::multiply_kernel(
  int LS_per_iteration, int size_reduction_block, int blocks_to_reduce)
{
  BEGIN_KERNEL("multiplyArrMatKernel");
    EXECUTE_IN_ALL(
      R12 = 0;                      // reset array LS index
    )

    for (int i = 0; i < LS_per_iteration; i++) {
      EXECUTE_IN_ALL(
        R1 = LS[R12];               // load input array
        R0 = INDEX;                 // Used later to select PEs for reduction
        R6 = size_reduction_block;  // Used to select blocks for reduction

        R3 = R1 * R2;               // a1 * a2, b1 * b2
        R3 = MULT_HIGH();

        CELL_SHL(R2, R11);          // Bring b2 to the left to calc b2 * a1
        NOP;
        R4 = SHIFT_REG;
        R4 = R1 * R4;               // a1 * b2
        R4 = MULT_HIGH();

        CELL_SHR(R2, R11);
        NOP;
        R5 = SHIFT_REG;
        R5 = R1 * R5;               // b1 * a2
        R5 = MULT_HIGH();

        R9 = R8 & R11;
        R7 = (R9 == R11);
        NOP;
      )

      EXECUTE_WHERE_EQ(             // Only in the odd PEs
        // Getting -b1 * b2 in each odd cell
        R3 = R10 - R3;              // All partial real parts are in R3

        R4 = R5;                    // All partial imaginary parts are now in R4
      )

      REPEAT_X_TIMES(blocks_to_reduce);
        EXECUTE_IN_ALL(
          R7 = (R0 < R6);           // Select only blocks of PEs at a time
          NOP;
        )
        EXECUTE_WHERE_LT(
          R0 = 129;                 // A random number > 128 so these PEs won't
                                    // be selected again
          REDUCE(R3);               // Real part
          REDUCE(R4);               // Imaginary part
        )
        EXECUTE_IN_ALL(
          R6 = R6 + R14;            // Go to the next block of PEs
        )
      END_REPEAT;

      EXECUTE_IN_ALL(
        R12 = R12 + R11;            // Go to the next LS
      )
    }
  END_KERNEL("multiplyArrMatKernel");
}
\end{lstlisting}

\section{Kernel pentru înmulțirea înlănțuită dintre un vector linie, o matrice
și un vector coloană}
\label{sec:kernel-mult-chained}

\lstset{
    language=C++,
    directivestyle={\color{black}},
    emph={int,char,double,float,unsigned},
}
\begin{lstlisting}[
    caption={Kernel pentru înmulțirea înlănțuită dintre un vector linie, o matrice
și un vector coloană în situația în care cel puțin o matrice poate fi scrisă pe
o linie din LS},
    label={lst:kernel-mult-chained}
]
void MUSIC_lin_array_cnx_impl::init_chained(int size_red_block, int LS_mat_)
{
  BEGIN_KERNEL("initKernelChained");
    EXECUTE_IN_ALL(
      R26 = LS_mat_;        // From where to start reading the matrix
      R30 = 1;
      R31 = 0;
      R28 = size_red_block; // dimension of the blocks on which reduction is
                            // performed at once
      R9 = INDEX;           // Select only the odd PEs
    )
  END_KERNEL("initKernelChained");
}

 void MUSIC_lin_array_cnx_impl::init_index_chained(void)
{
  BEGIN_KERNEL("initIndexChained");
    EXECUTE_IN_ALL(
      R2 = LS[R26];           // Load new matrix
    )
  END_KERNEL("initIndexChained");
}

void MUSIC_lin_array_cnx_impl::multiply_chained(int LS_per_execution,
  int size_red_block, int nr_red_blocks, int LS_in_arr_, int LS_final_real_,
  int LS_final_imag_)
{
  BEGIN_KERNEL("multChained");
    EXECUTE_IN_ALL(
      R25 = LS_in_arr_;       // Reset the array index
      R24 = LS_final_real_;   // Reset the real part of the final array index
      R23 = LS_final_imag_;   // Reset the imag part of the final array index
    )

    for (int i = 0; i < LS_per_execution; i++) {
      EXECUTE_IN_ALL(
        R1 = LS[R25];         // Load input array
        R15 = LS[R24];        // Load real part of the final array
        R16 = LS[R23];        // Load imag part of the final array
        R29 = INDEX;          // Used later to select PEs for reduction
        R27 = size_red_block; // Used to select blocks for reduction

        R16 = R31 - R16;      // Invert imag part of the final array

        R3 = R1 * R2;         // a1 * a2, b1 * b2
        R3 = MULT_HIGH();     // Get partial real parts of the first arr x
                              // mat product
        CELL_SHL(R1, R30);    // Bring b1 to the left to calc b1 * a2
        NOP;
        R4 = SHIFT_REG;
        R4 = R4 * R2;         // b1 * a2
        R4 = MULT_HIGH();

        CELL_SHR(R1, R30);
        NOP;
        R5 = SHIFT_REG;
        R5 = R5 * R2;         // b2 * a1
        R5 = MULT_HIGH();

        R10 = R9 & R30;
        R7 = (R10 == R30);
        NOP;
      )

      EXECUTE_WHERE_EQ(       // Only in the odd PEs
        R3 = R31 - R3;        // Invert sign of these partial real parts

        R4 = R5;              // All partial imaginary parts of the first
                              // product are now in R4
      )

      EXECUTE_IN_ALL(
        R3 = R3 * R15;         // Multiply by the real part of the elements
        R3 = MULT_HIGH();     // in the final array

        R4 = R4 * R16;         // Multiply the partial imag parts of the
                              // first product by the imag part of the final array
        R4 = MULT_HIGH();
      )

      // We reduce blocks of size mat_size_c; Half of the final result is in
      // R3 and the other half is in R4
      REPEAT_X_TIMES(nr_red_blocks);
        EXECUTE_IN_ALL(
          R7 = (R29 < R27);   // Select only blocks of PEs at a time
          NOP;
        )
        EXECUTE_WHERE_LT(
          R29 = 129;          // A random number > 128 so these PEs won't be
                              // selected again
          REDUCE(R3);         // Real part
          REDUCE(R4);         // Imaginary part
        )
        EXECUTE_IN_ALL(
          R27 = R27 + R28;    // Go to the next block of PEs
        )
      END_REPEAT;

      EXECUTE_IN_ALL(
        R25 = R25 + R30;      // Go to the next LS
        R24 = R24 + R30;
        R23 = R23 + R30;
      )
    }
  END_KERNEL("multChained");
}
\end{lstlisting}



\section{Kernel pentru realizarea autocorelației}
\label{sec:kernel-autocorr-code}

\lstset{
    language=C++,
    directivestyle={\color{black}},
    emph={int,char,double,float,unsigned},
}
\begin{lstlisting}[
    caption={Kernel pentru calculul autocorelației unui semnal},
    label={lst:kernel-autocorr-code}
]
void autocorrelate_cnx_impl::initKernel(const int &LS_per_row_) {
  BEGIN_KERNEL("initKernel");
    EXECUTE_IN_ALL(
      R0 = LS_per_row_;
      R29 = 1;
      R28 = 0;
      R25 = 2;
    )
  END_KERNEL("initKernel");
}

void autocorrelate_cnx_impl::autocorrelationKernel(
  const int &iterations_per_array)
{
  BEGIN_KERNEL("autocorrelationKernel");
    for (int i = 0; i < n_rows; i++) {
      for (int j = i; j < n_rows; j++) {
        EXECUTE_IN_ALL(
          R14 = i;
          R15 = j;

          R30 = R14 * R0;           // go to line i
          R31 = R15 * R0;           // go to col j
        )

        REPEAT_X_TIMES(iterations_per_array);
          EXECUTE_IN_ALL(
            R1 = LS[R30];
            R2 = LS[R31];
            R3 = R1 * R2;
            R3 = MULT_HIGH();       // re1 * re2, im1 * im2

            CELL_SHL(R2, R29);
            NOP;
            R4 = SHIFT_REG;
            R4 = R4 * R1;
            R4 = MULT_HIGH();       // re1 * im2
            R4 = R28 - R4;          // The column is conjugated => negate these

            CELL_SHR(R2, R29);
            NOP;
            R5 = SHIFT_REG;
            R5 = R5 * R1;
            R5 = MULT_HIGH();       // re2 * im1;

            R6 = INDEX;             // Select only the odd PEs
            R6 = R6 & R29;
            R7 = (R6 == R29);
            NOP;
          )

          EXECUTE_WHERE_EQ(
            R4 = R5;                // All partial imaginary parts are now in R4
          )

          EXECUTE_IN_ALL(
            REDUCE(R3);
            REDUCE(R4);

            R30 = R30 + R29;        // Go to next chunk on line & col
            R31 = R31 + R29;
          )
        END_REPEAT;
      }
    }
  END_KERNEL("autocorrelationKernel");
}
\end{lstlisting}

