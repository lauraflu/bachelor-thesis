\chapter{Testarea lanțului de procesare MUSIC}

\section{Declararea claselor necesare în testare}
\label{sec:test-header}

\lstset{
    basicstyle=\fontsize{9}{9}\selectfont\ttfamily\color{black},
    language=C++,
    directivestyle={\color{black}},
    emph={int,char,double,float,unsigned},
    emphstyle={\color{mygray}},
    numberstyle=\tiny\color{mygray},                                            
    keywordstyle=\color{black}\bfseries,                                               
    showspaces=false,                                                           
    showstringspaces=false,                                                     
    stringstyle=\color{mygray}  
}
\begin{lstlisting}[
    caption={Declararea claselor necesare în testare},
    label={lst:test-header}
]
#ifndef DOA_TEST_H_
#define DOA_TEST_H_

#include <array>
#include <string>
#include <vector>

#include <gnuradio/analog/sig_source_waveform.h>
#include <gnuradio/analog/noise_type.h>

// Used to access the command line arguments from inside the test cases
extern int my_argc;
extern char **my_argv;

template <int SIGNAL_SOURCES>
class flowgraph_parameters
{
    template <typename T>
    using array = std::array<T, SIGNAL_SOURCES>;

public:
    double sample_rate;
    float norm_spacing;
    int num_array_elements;
    int p_spectrum_length;
    int snapshot_size;
    int overlap_size;
    int nr_output_items;
    array<double> freq;
    array<int> theta_deg;
    array<double> signal_amplitude;
    array<double> noise_amplitude;
    array<gr::analog::gr_waveform_t> waveform;
    array<gr::analog::noise_type_t> noise_type;
    std::string distributionFIFO;
    std::string reductionFIFO;
    std::string writeFIFO;
    std::string readFIFO;

    flowgraph_parameters() {}

    flowgraph_parameters(flowgraph_parameters const &fg_param) :
            sample_rate(fg_param.sample_rate),
            norm_spacing(fg_param.norm_spacing),
            num_array_elements(fg_param.num_array_elements),
            p_spectrum_length(fg_param.p_spectrum_length),
            snapshot_size(fg_param.snapshot_size),
            overlap_size(fg_param.overlap_size),
            nr_output_items(fg_param.nr_output_items),
            freq(fg_param.freq),
            theta_deg(fg_param.theta_deg),
            signal_amplitude(fg_param.signal_amplitude),
            noise_amplitude(fg_param.noise_amplitude),
            waveform(fg_param.waveform),
            noise_type(fg_param.noise_type),
            distributionFIFO(fg_param.distributionFIFO),
            reductionFIFO(fg_param.reductionFIFO),
            writeFIFO(fg_param.writeFIFO),
            readFIFO(fg_param.readFIFO)
        {}

    flowgraph_parameters &set_sample_rate(double sample_rate_) {
        sample_rate = sample_rate_;
        return *this;
    }

    flowgraph_parameters &set_norm_spacing(float norm_spacing_) {
        norm_spacing = norm_spacing_;
        return *this;
    }

    flowgraph_parameters &set_num_array_elements(int num_array_elements_) {
        num_array_elements = num_array_elements_;
        return *this;
    }

    flowgraph_parameters &set_p_spectrum_length(int p_spectrum_length_) {
        p_spectrum_length = p_spectrum_length_;
        return *this;
    }

    flowgraph_parameters &set_snapshot_size(int snapshot_size_) {
        snapshot_size = snapshot_size_;
        return *this;
    }

    flowgraph_parameters &set_overlap_size(int overlap_size_) {
        overlap_size = overlap_size_;
        return *this;
    }

    flowgraph_parameters &set_nr_output_items(int nr_output_items_) {
        nr_output_items = nr_output_items_;
        return *this;
    }

    flowgraph_parameters &set_freq(const array<double> &freq_) {
        freq = freq_;
        return *this;
    }

    flowgraph_parameters
        &set_theta_deg(const array<int> &theta_deg_) {
        theta_deg = theta_deg_;
        return *this;
    }

    flowgraph_parameters &set_signal_amplitude(
                        const array<double> &signal_amplitude_) {
        signal_amplitude = signal_amplitude_;
        return *this;
    }

    flowgraph_parameters &set_noise_amplitude(
                        const array<double> &noise_amplitude_) {
        noise_amplitude = noise_amplitude_;
        return *this;
    }

    flowgraph_parameters &set_waveform(
                        const array<gr::analog::gr_waveform_t> &waveform_) {
        waveform = waveform_;
        return *this;
    }

    flowgraph_parameters &set_noise_type(
                        const array<gr::analog::noise_type_t> &noise_type_) {
        noise_type = noise_type_;
        return *this;
    }

    flowgraph_parameters
        &set_distributionFIFO(std::string distributionFIFO_) {
        distributionFIFO = distributionFIFO_;
        return *this;
    }

    flowgraph_parameters
        &set_reductionFIFO(std::string reductionFIFO_) {
        reductionFIFO = reductionFIFO_;
        return *this;
    }

    flowgraph_parameters
        &set_writeFIFO(std::string writeFIFO_) {
        writeFIFO = writeFIFO_;
        return *this;
    }

    flowgraph_parameters
        &set_readFIFO(std::string readFIFO_) {
        readFIFO = readFIFO_;
        return *this;
    }
};

template <int SIGNAL_SOURCES>
class doa_flowgraph
{
public:
    flowgraph_parameters<SIGNAL_SOURCES> fg_param;

    doa_flowgraph(const flowgraph_parameters<SIGNAL_SOURCES> &fg_param_) :
        fg_param(fg_param_)
        {}

    std::vector<float> build_flowgraph();
};
#endif // DOA_TEST_H_
\end{lstlisting}

\section{Definirea metodei de construcție a unui lanț de procesare}
\label{sec:test-code}

\lstset{
    language=C++,
    directivestyle={\color{black}},
    emph={int,char,double,float,unsigned},
}
\begin{lstlisting}[
    caption={Definirea metodei de construcție a unui lanț de procesare},
    label={lst:test-code}
]
#include "doa_test.h"

#include <gnuradio/top_block.h>
#include <gnuradio/analog/sig_source_c.h>
#include <gnuradio/analog/noise_source_c.h>
#include <gnuradio/blocks/add_cc.h>
#include <gnuradio/blocks/throttle.h>
#include <gnuradio/blocks/multiply_matrix_cc.h>
#include <gnuradio/blocks/null_sink.h>
#include <gnuradio/blocks/vector_to_streams.h>
#include <gnuradio/blocks/head.h>
#include <gnuradio/blocks/vector_sink_f.h>
#include <gnuradio/gr_complex.h>

#include <doa/autocorrelateConnexKernel.h>
#include <doa/autocorrelate.h>
#include <doa/MUSIC_lin_array.h>
#include <doa/MUSIC_lin_array_cnx.h>
#include <doa/find_local_max.h>

#include <boost/math/constants/constants.hpp> // for type float PI

#include <cmath>
#include <complex>
#include <chrono>
#include <thread>

using namespace std::complex_literals;

template class doa_flowgraph<2>;

template <int SIGNAL_SOURCES>
std::vector<float> doa_flowgraph<SIGNAL_SOURCES>::build_flowgraph()
{
    gr::top_block_sptr tb = gr::make_top_block("run_MUSIC");

    const float pi = boost::math::constants::pi<float>();
    const int num_targets = 2;

    float theta0 = pi * fg_param.theta_deg[0] / 180;
    float theta1 = pi * fg_param.theta_deg[1] / 180;

    std::vector<gr_complex> amv0(fg_param.num_array_elements);
    std::vector<gr_complex> amv1(fg_param.num_array_elements);
    std::vector<std::vector<gr_complex>> array_manifold_matrix(
        fg_param.num_array_elements,
        std::vector<gr_complex>(num_targets)
    );

    float ant_loc_val;

    if (fg_param.num_array_elements % 2 == 1) {
        ant_loc_val = (float)fg_param.num_array_elements / 2;
    } else {
        ant_loc_val = (float)fg_param.num_array_elements / 2 - 0.5;
    }

    // can't multiply by an integer (2) when we have -1if so it's a workaround
    float temp_norm_spacing = fg_param.norm_spacing * 2;
    for (int i = 0; i < fg_param.num_array_elements; i++) {
        amv0[i] = std::exp(-1if * temp_norm_spacing * pi * ant_loc_val *
                            cosf(theta0));
        amv1[i] = std::exp(-1if * temp_norm_spacing * pi * ant_loc_val *
                            cosf(theta1));
        array_manifold_matrix[i][0] = amv0[i];
        array_manifold_matrix[i][1] = amv1[i];

        ant_loc_val--;
    }

    /*=============================================
     * Blocks
     *============================================*/

    gr::analog::sig_source_c::sptr sig_src0 = gr::analog::sig_source_c::make(
    fg_param.sample_rate,               // sample rate
    fg_param.waveform[0],               // waveform used
        fg_param.freq[0],               // wave frequency
        fg_param.signal_amplitude[0],   // amplitude
        0                               // offset
    );

    gr::analog::sig_source_c::sptr sig_src1 = gr::analog::sig_source_c::make(
        fg_param.sample_rate,           // sample rate
        fg_param.waveform[1],           // waveform used
        fg_param.freq[1],               // wave frequency
        fg_param.signal_amplitude[1],   // amplitude
        0                               // offset
    );

    gr::analog::noise_source_c::sptr n_src0 =
    gr::analog::noise_source_c::make(
        fg_param.noise_type[0],         // type of noise
        fg_param.noise_amplitude[0]     // amplitude of the noise
    );

    gr::analog::noise_source_c::sptr n_src1 =
    gr::analog::noise_source_c::make(
        fg_param.noise_type[1],         // type of noise
        fg_param.noise_amplitude[1]     // amplitude of the noise
    );

    gr::blocks::throttle::sptr thr0 = gr::blocks::throttle::make(
        sizeof(gr_complex),             // item size
        fg_param.sample_rate            // sample rate
    );

    gr::blocks::add_cc::sptr add0 = gr::blocks::add_cc::make();

    gr::blocks::add_cc::sptr add1 = gr::blocks::add_cc::make();

    gr::blocks::multiply_matrix_cc::sptr mult_by_matrix0 =
        gr::blocks::multiply_matrix_cc::make(
            array_manifold_matrix,
            gr::block::TPP_ALL_TO_ALL
        );

    gr::doa::autocorrelate::sptr autocorr0 =
      gr::doa::autocorrelate::make(
            fg_param.num_array_elements,
            fg_param.snapshot_size,
            fg_param.overlap_size,
            1
        );

    gr::doa::MUSIC_lin_array::sptr music_lin_array0 =
        gr::doa::MUSIC_lin_array::make(
            fg_param.norm_spacing,
            num_targets,
            fg_param.num_array_elements,
            fg_param.p_spectrum_length
        );

    gr::doa::find_local_max::sptr find_local_max0 =
        gr::doa::find_local_max::make(
            num_targets,        // maximum number of local maximums
            fg_param.p_spectrum_length,
            0.0,                // min value of index vector
            180.0               // max value of index vector
        );

    // Null sink for locations
    gr::blocks::null_sink::sptr null_sink0 =
        gr::blocks::null_sink::make(
            sizeof(float) * num_targets // item size
        );

    gr::blocks::vector_to_streams::sptr v_to_s0 =
        gr::blocks::vector_to_streams::make(
            sizeof(float),              // item size
            num_targets                 // number of streams
        );

    // Limits the number of output items
    gr::blocks::head::sptr head0 = gr::blocks::head::make(
        sizeof(float),                  // size of stream item
        fg_param.nr_output_items        // number of output items
    );
    gr::blocks::head::sptr head1 = gr::blocks::head::make(
        sizeof(float),                  // size of stream item
        fg_param.nr_output_items        // number of output items
    );

    gr::blocks::vector_sink_f::sptr v_sink0 = gr::blocks::vector_sink_f::make();
    gr::blocks::vector_sink_f::sptr v_sink1 = gr::blocks::vector_sink_f::make();

    /*=============================================
     * Connections
     *============================================*/

    tb->connect(sig_src0, 0, add0, 0);
    tb->connect(sig_src1, 0, add1, 0);

    tb->connect(n_src0, 0, add0, 1);
    tb->connect(n_src1, 0, thr0, 0);
    tb->connect(thr0, 0, add1, 1);

    tb->connect(add0, 0, mult_by_matrix0, 0);
    tb->connect(add1, 0, mult_by_matrix0, 1);

    for (int i = 0; i < fg_param.num_array_elements; i++) {
      tb->connect(mult_by_matrix0, i, autocorr0, i);
    }

    tb->connect(autocorr0, 0, music_lin_array0, 0);
    tb->connect(music_lin_array0, 0, find_local_max0, 0);
    tb->connect(find_local_max0, 0, null_sink0, 0);

    tb->connect(find_local_max0, 1, v_to_s0, 0);
    tb->connect(v_to_s0, 0, head0, 0);
    tb->connect(v_to_s0, 1, head1, 0);

    tb->connect(head0, 0, v_sink0, 0);
    tb->connect(head1, 0, v_sink1, 0);

    /*=============================================
     * Run the graph
     *============================================*/
    tb->start();

    std::thread t([&tb]{
        // Let run for 5 seconds
        std::this_thread::sleep_for(std::chrono::seconds(5));
        tb->stop();
        tb->wait();
    });
    t.join();

    std::vector<float> out_data0 = v_sink0->data();
    std::vector<float> out_data1 = v_sink1->data();

    if ((out_data0.size() > 0) && (out_data1.size() > 0))
    {
        size_t last_idx0 = out_data0.size() - 1;
        size_t last_idx1 = out_data1.size() - 1;

        std::cout << "Processed samples: " << out_data0.size()  << std::endl;

        std::vector<float> out_data;
        out_data.push_back(out_data0[last_idx0]);
        out_data.push_back(out_data1[last_idx1]);
        return out_data;
    } else {
        return {-1, -1};
    }
}
\end{lstlisting}


