\chapter{Concluzii}
\label{chapter:concl}

Rezultatele obținute în privința performanțelor blocurilor ce utilizează
kerneluri ConnexArray demonstrează faptul că acestea pot fi folosite în
accelerarea algoritmului de localizare MUSIC pentru un plus de performanță cu un
consum de putere redus. Integrarea acestora în blocuri din mediul
GNU Radio pentru radio definit software le facilitează utilizarea în aplicații
deja existente și le extinde domeniul de aplicabilitate, în contextul unei
dezvoltări accentuate a acestui tip de aplicații. \\

În plus față de implementarea propriu-zisă a blocurilor accelerate folosind
ansamblul Connex-ARM, am creat și o metodă generică de testare cu mediul
Google Test, care curpinde peste 100 de teste de evaluare a rezoluției și
preciziei obținute în diverse configurații ale sistemului, care poate fi
utilizată inclusiv cu simulatorul OPINCAA și sistemul hardware. \\

Evaluarea performanțelor atinse de implementarea algoritmului MUSIC, atât în
varianta originală, cât și în cea accelerată, deschide mai multe posibilități de
îmbunătățire a acestora. Acest aspect devine important dacă luăm în considerare
faptul că, deși algoritmul MUSIC este preferat din motive de precizie,
stabilitate și rezoluție, el oferă aceste avantaje cu costul creșterii
resurselor computaționale necesare în implementarea sa și este de dorit să se
găsească metode prin care acestea pot fi reduse. \\

Una dintre aceste posibilități este legată de cerința folosirii acceleratorului
ConnexArray de mai multe blocuri de procesare, care se poate dovedi extrem de
utilă în aplicații radio definite software în care procesarea este distribuită
în mai multe blocuri cu sarcini bine definite și vom considera această temă o
posibilă direcție de cercetare în viitor.

