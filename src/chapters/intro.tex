\chapter{Introducere}
\label{chapter:intro}

Sistemele radio definite software (SDR - Software Defined Radio), în care
funcții din nivelul fizic al echipamentelor sunt implementate software, a prins
amploare odată cu creșterea capabilității procesoarelor de uz general, astfel
încât, în mod curent, toate echipamentele radio au implementat într-o oarecare
măsură acest concept. Migrarea către software nu este deloc surprinzătoare:
echipamentele devin mai ușor (re)configurabile, actualizările se efectuează rapid, avantaj
deosebit de important în contextul evoluției rapide a tehnologiei, iar costurile
scad, deoarece nu se mai impune înlocuirea unor componente \textit{hardware} care nu mai
corespund cerințelor pieței. \\

Deși procesoarele de uz general s-au adaptat cerințelor de procesare impuse de
procesarea digitală a semnalelor prin implementarea unor extensii multimedia
pentru instrucțiuni SIMD, acestea sunt limitate din punct de vedere al
capacității oferite, iar compilatoarele adesea nu pot efectua o vectorizare
eficientă a codului și în unele cazuri nu oferă deloc suport pentru instrucțiuni
SIMD. Dorim să găsim o alternativă pentru procesarea digitală a semnalelor mai
eficientă folosind suportul unui procesor, fără a sacrifica
programabilitatea atât de importantă în SDR. \\

Propunem folosirea unui ansamblu procesor-accelerator cu suport pentru
programarea acceleratorului în interiorul codului executat pe procesor, care va
efectua funcții de control, lăsând în sarcina acceleratorului părțile intensive
din punct de vedere computațional care ar putea beneficia de suport SIMD.
Ansamblul reprezintă o cale de mijloc între resursele de procesare insuficiente
oferite de către extensiile multimedia și programarea dificilă a sistemelor care
decuplează total procesorul gazdă de un accelerator adaptat perfect cerințelor
aplicației pentru care este folosit. Sistemul utilizat este denumit
Connex-ARM~\cite{energy-effective-simd} și este implementat pe platforma Xilinx
Zynq-7000~\cite{zynq}, în al cărei chip FPGA se află acceleratorul cu
arhitectura ConnexArray. \\

Vom analiza performanțele sistemului obținute în localizarea sursei unui semnal
radio, în care anumite puncte critice din lanțul de procesare vor fi
implementate pe accelerator. Algoritmul folosit pentru localizarea sursei unui
semnal este MUSIC~\cite{music-schmidt} (MUltiple SIgnal Classification), iar implementarea folosită
ca referință~\cite{ettus-doa} este integrată în mediul GNU
Radio~\cite{gnuradio}, dedicat aplicațiilor SDR. Ansamblul folosit prezintă, în
plus, o alternativă eficientă din punct de vedere energetic~\cite{fpga-visual},
aspect deosebit de important în echipamentele de telecomunicații.

Proiectul cuprinde, astfel, următoarele etape:
\begin{itemize}
  \item Studiul algoritmului folosit și al tehnologiilor folosite în proiect, a
  căror funcționare va fi sintetizată în Capitolul~\ref{chapter:background}.

  \item Identificarea punctelor critice din lanțul de procesare care
  implementează algoritmul MUSIC, care va fi făcută în
  Capitolul~\ref{chapter:impl}, alături de o descriere a funcționării lanțului
  de procesare și a metodologiei utilizate în realizarea profilului său de execuție.

  \item Implementarea unor \textit{kerneluri} pentru acceleratorul ConnexArray
  corespunzătoare punctelor critice identificate, detaliate în
  Capitolul~\ref{chapter:acc}. În acest capitol se vor descrie și pașii
  efectuați în integrarea kernelurilor în mediul GNU Radio, integrate în blocuri de sine
  stătătoare corespunzătoare unei funcționalități din lanțul {MUSIC}.

  \item Evaluarea performanțelor blocurilor nou create în comparație cu cele
  inițiale atât separat, cât și în întregul lanț de procesarei va fi descrisă în
  Capitolul~\ref{chapter:eval}.
\end{itemize}

În finalul lucrării, în Capitolul \ref{chapter:concl}, vom prezenta concluziile
proiectului, precum și diverse îmbunătățiri care îi vor fi aduse pe viitor.
