%============================================================================
% ADD SUPPORT FOR OPINCAA IN GNU RADIO
%============================================================================

\section{Add support for OPINCAA in GNU Radio}
\label{sec:opincaa}

Since the ConnexArray is made to work with vectors, it does not make much sense
to add only two numbers as before, so instead we have two input arrays whose
elements are loaded into the PEs, then elements with the same index from the two
arrays are added into the PEs and, finally, a reduction is performed over all
the active PEs, so the result is basically the sum of all the elements in the
input arrays.\\

The following changes need to be made to the block:
\begin{itemize}
  \item I chose to keep a pointer to the \code{ConnexMachine} object as a
          private member of the \code{add\_ii} block. Then, since we need only an
          instance of it, we create it in the constructor of the class
          \S\ref{ssec:ctor}, using the arguments that contain the paths to the FIFO queues.
          
  \item In the worker, we define a simple \code{add} kernel that initializes R1
          and R2 with the contents of, respectively, the first and second vectors of
          the local storage, then adds their contents and stores the result in R2.
          Finally, a \code{reduce} operation is applied to all the R2 registers.

  \item Before executing the kernel, we write the two input arrays to the first two
          vectors of the local storage using the method
          \code{writeDataToArray}, which takes as arguments, respectively, the
          array to be written, its number of elements and the vector (the local
          storage location) in which the elements are stored.
          Finally, we run the kernel and assign the result of the reduction to the
          first element of the output vector. The result represents the sum of
          all the numbers in the arrays.
\end{itemize}

The full source code of this function can be found in section
\S\ref{ssec:work-kernel}.\\

There were several problems at compile time. Firstly, the
\code{ConnexMachine.h} header needs to be added to the
\textbf{add\_ii\_impl.h} file. Because this header uses certain features from
C++11, support for this standard must be added, so the line

\begin{lstlisting}
SET(CMAKE_CXX_FLAGS "\${CMAKE_CXX_FLAGS} -std=c++0x")
\end{lstlisting}

needs to be written in the \textbf{CMakeLists.txt} file, in the \textbf{Compiler
Specific Setup} section.\\

Then, the \code{opincaa} library needs to be added to the
\code{GR\_SWIG\_LIBRARIES} in \textbf{swig/CMakeLists.txt}, in order to be
able to link the symbols related to it.\\

Even with these changes, I was unable to use the Python unit testing, as even
though the ConnexMachine was created and the kernel was succesfully executed,
giving correct results, right before returning the \code{noutput\_items} in
the \code{work} method, the thread caught an unrecognized exception. I suspect
the problem comes from the way in which SWIG connects the C++ modules with the
Python tests and this is an issue I will further investigate, as I would like
to be able to use the unit testing framework with Python.\\

Given this issue, I chose to create a C++ test to verify if I would encounter the
same problem. I used a similar stucture of the testing code as I did in the
Python test, as can be seen in the subsection \S\ref{ssec:test-cpp}. \\

The next step was to create a Makefile to build the block, as well as the test
( \S\ref{ssec:makefile-test-cpp} ). Also, in order to make the testing easier, I
created a bash script to run the simulator and the test with the correct
parameters. It is worthwhile mentioning that, currently, the OPINCAA project
must be installed separately (and the simulator for the ConnexArray SIMD must
be also built along) and the relative (or absolute) path to the project must
be set before running the test. A better approach would be adding the OPINCAA
project directly as a git submodule for the current module, in order to ensure
that all the dependencies are satisfied and to eliminate the need to modify
the path. \\

By running \code{make} and then \code{./test\_simulator.sh} I get the following
output:\\

\begin{lstlisting}[language=bash]
laoo@neverland ~/work/licenta/gr-opincaa $ ./test_simulator.sh 
FIFO distributionFIFO succesfully opened!
FIFO writeFIFO succesfully opened!
FIFO reductionFIFO succesfully opened!
FIFO readFIFO succesfully opened!
Starting Core Thread...
Starting IO Thread...
/home/laoo/work/licenta/gr-opincaa/distributionFIFO
/home/laoo/work/licenta/gr-opincaa/reductionFIFO
/home/laoo/work/licenta/gr-opincaa/writeFIFO
/home/laoo/work/licenta/gr-opincaa/readFIFO
ConnexMachine created !
Reduction = 105
./test_simulator.sh: line 38: 13175 Killed
\end{lstlisting}

The sum of all the elements in the two input arrays is, indeed, 105, so the
block is functioning correctly.
