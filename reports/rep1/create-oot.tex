%============================================================================
% CREATE OOT MODULE
%============================================================================

\section{Steps taken to create a new out-of-tree module}
\label{sec:steps-oot}
\begin{itemize}
	\item Used the \textbf{gr\_modtool} script to create a new module named
		\textbf{gr-opincaa}. This creates the directory structure and some of the
		files specified above, using the following command:\\
	
		\begin{lstlisting}[language=bash]
gr_modtool newmod opincaa
		\end{lstlisting}
	
	\item Created a block that adds two integers on the Connex SIMD, named
		\textbf{add\_ii}. For this step, we also use \textbf{gr\_modtool},
		specifying it to create a new module which has a 1:1 relationship between
		the input rate and the output rate, specified by the parameter \code{sync}:
	
		\begin{lstlisting}[language=bash]
gr_modtool add -t sync add_ii
		\end{lstlisting}
	
		Additionally, I added Python QA support to be able to create tests for the block.
		Also, I foresaw a step needed for creating a ConnexMachine and when asked to
		"Enter valid argument list, including default arguments:",
		I specified the following parameters:\\
		\code{std::string distributionFIFO},
		\code{std::string reductionFIFO}, \\
		\code{std::string writeFIFO} and
		\code{std::string readFIFO}.

		This is because the ConnexMachine object needs to receive as parameters the
		FIFO queues that help it to communicate with the SIMD processor or the
		simulator. Since we pass the paths as parameters, we do not need to take
		care of both of the cases in the worker and, should the paths ever change,
		we need only to call the tester with the right arguments.

	\item The \textbf{add\_ii.cc} file needs to be modified to fit with its
		purpose. Since we want to implement an adder, we need two input ports and
		one output port, so the constructor of the block needs to have the following
		structure:\\
	
		\lstset{
			language=C++,
			directivestyle={\color{black}},
			emph={int,char,double,float,unsigned},
			emphstyle={\color{RoyalBlue}}
		}
		\begin{minipage}{\linewidth}
		\begin{lstlisting}
add_ii_impl::add_ii_impl(
		std::string distributionFIFO,                                           
		std::string reductionFIFO,                                              
		std::string writeFIFO,                                                  
		std::string readFIFO          
		)                                                
	: gr::sync_block
	("add_ii",                                               
	  gr::io_signature::make(2, 2, sizeof(int)),                           
	  gr::io_signature::make(1, 1, sizeof(int)))                           
{} 
		\end{lstlisting}
	\end{minipage}

		The \code{io\_signature} method takes information about the minimum
		number of ports, the maximum number of ports and their size,
		respectively. Also, we notice that \textbf{gr\_modtool} added the specified 
		parameters as arguments to the constructor.


	\item Fot the beginning, in the program I simply added two numbers using C++
		and wrote a test for it, to make sure that it is working accordingly. The
		processing is done in the worker, where \code{input\_items[i]} and
		\code{output\_items[i]} are arrays which contains the data of the
		\code{i}-th input and output, respectively.

		When using the \code{work} method, the length of the input arrays is
		equal with the one of the output, so we iterate over
		\code{noutput\_items} to compute the sum. The full code of the worker 
		function can be found in the subsection \S\ref{ssec:simple-worker}.
	
	
	\item Since we want to ensure that the program is running correctly, we must 
		write a test for it. The code can be found in the subsection \S\ref{ssec:python-test}
	
		We define two input arrays and another one that contains the output we would 
		expect. Then, we source the elements of both of the inputs, state that 
		\code{opincaa.add\_ii} is the block that we are testing and gather its output 
		with \code{blocks.vector\_sink\_i}. Afterwards we build the graph, run it 
		and compare the results.

		\begin{lstlisting}
Test project /home/laoo/work/licenta/gr-opincaa/build
    Start 3: qa_add_ii
1/1 Test \#3: qa_add_ii ............. Passed    0.16 sec

100% tests passed, 0 tests failed out of 1

Total Test time (real) =   0.16 sec
		\end{lstlisting}

		Since the test passes, we can proceed to add the OPINCAA kernel.
\end{itemize}
