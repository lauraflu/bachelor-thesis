%============================================================================
% CODE SNIPPETS
%============================================================================

\newpage
\section{Code snippets}
\label{sec:acc-code}

	\subsection{Accelerating the matrix multiplication - multithreading}
	\label{ssec:acc-mt}
		\lstset{
		    language=C++,
		    directivestyle={\color{black}},
		    emph={int,char,double,float,unsigned},
		    emphstyle={\color{RoyalBlue}}
		}		
		\begin{lstlisting}[
		    caption={The C++ code used for accelerating the matrix
                    multiplication in the MUSIC Linear Array block - using a
                    separate thread for launching the ConnexArray job},
		    label={lst:acc-mt}
		]
#ifdef HAVE_CONFIG_H
#include "config.h"
#endif

#include <gnuradio/io_signature.h>
#include "multiply_cc_impl.h"
#include <cmath>
#include <thread>

namespace gr {
  namespace opincaa {

    /*
     * Kernel related functions
     */
    void executeMultiplyArrMat(ConnexMachine *connex);
    void multiply_kernel(
      int process_at_once,
      int size_of_block,
      int blocks_to_reduce);
    void multiply_once(int size_of_block, int blocks_to_reduce);

    multiply_cc::sptr
    multiply_cc::make(
      std::string distributionFIFO,
      std::string reductionFIFO,
      std::string writeFIFO,
      std::string readFIFO)
    {
      return gnuradio::get_initial_sptr
        (new multiply_cc_impl(distributionFIFO,
                              reductionFIFO,
                              writeFIFO,
                              readFIFO));
    }

    /*
     * The private constructor
     */
    multiply_cc_impl::multiply_cc_impl(
      std::string distributionFIFO,
      std::string reductionFIFO,
      std::string writeFIFO,
      std::string readFIFO)
      : gr::sync_block("multiply_cc",
              gr::io_signature::make(2, 2, sizeof(gr_complex) * MAT_SIZE),
              gr::io_signature::make(1, 1, sizeof(gr_complex) * NR_ARRAYS_ELEMS))
    {
      try {
        connex = new ConnexMachine(distributionFIFO,
                                   reductionFIFO,
                                   writeFIFO,
                                   readFIFO);
      } catch (std::string err) {
        std::cout << err << std::endl;
      }

      const int blocks_to_reduce = VECTOR_ARRAY_SIZE / ARR_SIZE_C;
      const int size_of_block = ARR_SIZE_C;

      factor_mult1 = 1 << 14;
      factor_mult2 = 1 << 16;
      factor_res = 1 << 14;

      multiply_kernel(PROCESS_AT_ONCE, size_of_block, blocks_to_reduce);
    }

    /*
     * Our virtual destructor.
     */
    multiply_cc_impl::~multiply_cc_impl()
    {
      delete connex;
    }

    void multiply_cc_impl::forecast(
      int noutput_items,
      gr_vector_int &ninput_items_required)
    {
      // The first input is an array of ARR_SIZE elements, while the second is
      // a linearized, column by column matrix of ARR_SIZE x ARR_SIZE elements
      // The result is a 4 array element representing the multiplication between
      // the input array and input matrix.
      ninput_items_required[0] = (NR_ARRAYS / MAT_SIZE) * noutput_items;
      ninput_items_required[1] = noutput_items;
    }

    int
    multiply_cc_impl::work(int noutput_items,
        gr_vector_const_void_star &input_items,
        gr_vector_void_star &output_items)
    {
      if (!connex) {
        return noutput_items;
      }

      const gr_complex *in0 = reinterpret_cast<const gr_complex *>(input_items[0]);
      const gr_complex *in1 = reinterpret_cast<const gr_complex *>(input_items[1]);
      gr_complex *out = reinterpret_cast<gr_complex *>(output_items[0]);

      std::cout << "Called worker with " << noutput_items << " out items." << std::endl;

      for (int i = 0; i < noutput_items; i++) {
        std::cout << "Output item nr. " << i << std::endl;
        const gr_complex *in0_round = &in0[i * NR_ARRAYS_ELEMS];
        const gr_complex *in1_round = &in1[i * MAT_SIZE];
        gr_complex *out_round = &out[i * NR_ARRAYS_ELEMS];

        std::vector<gr_complex> final_res(NR_ARRAYS, 0);

        const int nr_chunks = NR_ARRAYS / ARR_IN_CHUNK;
        const int nr_elem_chunk = ARR_IN_CHUNK * ARR_SIZE;
        const int nr_elems_calc = PROCESS_AT_ONCE * VECTOR_ARRAY_SIZE / ARR_SIZE_C;

        uint16_t *in0_i = (uint16_t *)
            malloc(nr_chunks * VECTOR_ARRAY_SIZE * sizeof(uint16_t));
        uint16_t *in1_i = (uint16_t *)
            malloc(VECTOR_ARRAY_SIZE * sizeof(uint16_t));
        int32_t *res_mult = (int32_t *)
              malloc(nr_chunks * VECTOR_ARRAY_SIZE * sizeof(int32_t));

        if ((in0_i == NULL) || (in1_i == NULL) || (res_mult == NULL))
          std::cout << "Malloc error!" << std::endl;

        uint16_t *in0_curr, *in1_curr, *in0_next;

        // Prepare the first chunk
        in0_curr = in0_i;
        in1_curr = in1_i;

        int32_t *res_curr_chunk, *res_past_chunk = NULL;
        const gr_complex *arr_next_chunk, *arr_past_chunk;

        const gr_complex *arr_curr_chunk = in0_round;
        const gr_complex *mat = in1_round; // Using the same mat for all chunks
        gr_complex *out_curr_chunk = out_round, *out_past_chunk;

        const int elems_to_prepare = PROCESS_AT_ONCE * VECTOR_ARRAY_SIZE / 2;

        prepareInArrConnex(in0_curr, arr_curr_chunk, elems_to_prepare);
        prepareInMatConnex(in1_curr, mat, elems_to_prepare);

        for (int cnt_chunk = 0; cnt_chunk < nr_chunks; cnt_chunk++) {
          res_curr_chunk = &res_mult[cnt_chunk * VECTOR_ARRAY_SIZE];

          connex->writeDataToArray(in0_curr, PROCESS_AT_ONCE, 0);
          connex->writeDataToArray(in1_curr, PROCESS_AT_ONCE, 511);

          std::thread t(executeMultiplyArrMat, connex);

          try {
            // Prepare future data for all but the last chunk
            if (cnt_chunk != nr_chunks - 1) {
              in0_next = in0_curr + VECTOR_ARRAY_SIZE;
              arr_next_chunk = arr_curr_chunk + nr_elem_chunk;

              prepareInArrConnex(in0_next, arr_next_chunk, elems_to_prepare);
            }

            // Process past data for all but the first chunk
            if (cnt_chunk != 0) {
              out_past_chunk = &out_round[(cnt_chunk - 1) * nr_elem_chunk];
              prepareOutDataConnex(out_past_chunk, res_past_chunk, nr_elems_calc);
              multLineCol(final_res, out_past_chunk, arr_past_chunk, cnt_chunk - 1);
            }
          } catch (...) {
            t.join();
            throw;
          }

          // Must join threads before reading the reduction
          t.join();

          // 2 * ---> complex elements, so we have real *and* imag parts
          connex->readMultiReduction(2 * nr_elems_calc, res_curr_chunk);

          // Incrementing the pointers for the next chunk
          in0_curr = in0_next;
          arr_past_chunk = arr_curr_chunk;
          arr_curr_chunk = arr_next_chunk;
          res_past_chunk = res_curr_chunk;
        } // end loop for each chunk

        // Results for the last chunk
        out_past_chunk = &out_round[(nr_chunks - 1) * nr_elem_chunk];
        prepareOutDataConnex(out_past_chunk, res_past_chunk, nr_elems_calc);
        multLineCol(final_res, out_past_chunk, arr_past_chunk, nr_chunks - 1);

        free(in0_i);
        free(in1_i);
        free(res_mult);
      } // end loop for each output item

      return noutput_items;
    }

    void multiply_cc_impl::prepareInArrConnex(
          uint16_t *out_arr, const gr_complex *in_arr, const int nr_elems)
    {
      for (int j = 0; j < nr_elems; j++) {
        out_arr[2 * j] = static_cast<uint16_t>
          (in_arr[(j / 16) * ARR_SIZE + (j % 4)].real() * factor_mult1);
        out_arr[2 * j + 1] = static_cast<uint16_t>
          (in_arr[(j / 16) * ARR_SIZE + (j % 4)].imag() * factor_mult1);
      }
    }

    void multiply_cc_impl::prepareInMatConnex(
          uint16_t *out_mat, const gr_complex *in_mat, const int nr_elems)
    {
      for (int j = 0; j < nr_elems; j++) {
        out_mat[2 * j] = static_cast<uint16_t>
          (in_mat[j % 16].real() * factor_mult2);
        out_mat[2 * j + 1] = static_cast<uint16_t>
          (in_mat[j % 16].imag() * factor_mult2);
      }
    }

    void multiply_cc_impl::prepareOutDataConnex(
          gr_complex *out_data,
          const int32_t *raw_out_data,
          const int nr_elems)
    {
      float temp0, temp1;
      for (int j = 0; j < nr_elems; j++) {
        temp0 = (static_cast<float>(raw_out_data[j * 2])) / factor_res;
        temp1 = (static_cast<float>(raw_out_data[j * 2 + 1])) / factor_res;

        out_data[j] = std::complex<float>(temp0, temp1);
      }
    }

    void multiply_cc_impl::multLineCol(
          std::vector<gr_complex> result,
          const gr_complex *line_arr,
          const gr_complex *col_arr,
          const int nr_chunk)
    {
      // Do the final line array * column array multiplication
      // As a result, we have an element for each array processed in a chunk
      std::complex<float> acc(0, 0);
      for (int j = 0; j < ARR_IN_CHUNK; j++) {
        acc = (0, 0);
        for (int jj = 0; jj < ARR_SIZE; jj++) {
          acc = acc +
            (line_arr[j * ARR_IN_CHUNK + jj] * col_arr[j * ARR_IN_CHUNK + jj]);
        }
        result[nr_chunk * ARR_IN_CHUNK + j] = acc;
        std::cout << "result[" << nr_chunk * ARR_IN_CHUNK + j << "] = "
          << result[nr_chunk * ARR_IN_CHUNK + j] << std::endl;
      }
    }

    /*===================================================================
     * Define ConnexArray kernels that will be used in the worker
     *===================================================================*/
    void executeMultiplyArrMat(ConnexMachine *connex)
    {
      connex->executeKernel("multiply_arr_mat");
    }

    void multiply_kernel(int process_at_once, int size_of_block, int blocks_to_reduce)
    {
      BEGIN_KERNEL("multiply_arr_mat");
        EXECUTE_IN_ALL(
          R25 = 0;
          R26 = 511;
          R30 = 1;
          R31 = 0;
          R28 = size_of_block;  // Equal to ARR_SIZE_C; dimension of the blocks
                                // on which reduction is performed at once
        )

        EXECUTE_IN_ALL(
          R1 = LS[R25];           // z1 = a1 + j * b1
          R2 = LS[R26];           // z2 = a2 + j * b2
          R29 = INDEX;          // Used later to select PEs for reduction
          R27 = size_of_block;  // Used to select blocks of ARR_SIZE_C for reduction

          R3 = R1 * R2;         // a1 * a2, b1 * b2
          R3 = MULT_HIGH();

          CELL_SHL(R2, R30);    // Bring b2 to the left to calc b2 * a1
          NOP;
          R4 = SHIFT_REG;
          R4 = R1 * R4;         // a1 * b2
          R4 = MULT_HIGH();

          CELL_SHR(R2, R30);
          NOP;
          R5 = SHIFT_REG;
          R5 = R1 * R5;         // b1 * a2
          R5 = MULT_HIGH();

          R9 = INDEX;           // Select only the odd PEs
          R9 = R9 & R30;
          R7 = (R9 == R30);
        )

        EXECUTE_WHERE_EQ(       // Only in the odd PEs
          // Getting -b1 * b2 in each odd cell
          R3 = R31 - R3;        // All partial real parts are in R3

          R4 = R5;              // All partial imaginary parts are now in R4
        )

        REPEAT_X_TIMES(blocks_to_reduce);
          EXECUTE_IN_ALL(
            R7 = (R29 < R27);   // Select only blocks of 8 PEs at a time by
                                // checking that the index is < k * 8
          )
          EXECUTE_WHERE_LT(
            R29 = 129;          // A random number > 128 so these PEs won't be
                                // selected again
            REDUCE(R3);         // Real part
            REDUCE(R4);         // Imaginary part
          )
          EXECUTE_IN_ALL(
            R27 = R27 + R28;    // Go to the next block of 8 PEs
          )
        END_REPEAT;

      END_KERNEL("multiply_arr_mat");
    }
  } /* namespace opincaa */
} /* namespace gr */
                \end{lstlisting}

		\lstset{
		    language=C++,
		    directivestyle={\color{black}},
		    emph={int,char,double,float,unsigned},
		    emphstyle={\color{RoyalBlue}}
		}		
		\begin{lstlisting}[
		    caption={The C++ header for Listing \ref{lst:acc-mt}},
		    label={lst:acc-h-mt}
		]
#ifndef INCLUDED_OPINCAA_MULTIPLY_CC_IMPL_H
#define INCLUDED_OPINCAA_MULTIPLY_CC_IMPL_H

#include <opincaa/multiply_cc.h>
#include "ConnexMachine.h"

#define VECTOR_ARRAY_SIZE 128
#define ARR_SIZE 4
#define MAT_SIZE (ARR_SIZE * ARR_SIZE)

#define ARR_SIZE_C (ARR_SIZE * 2)
#define MAT_SIZE_C (MAT_SIZE * 2)

#define PROCESS_AT_ONCE 1

// Defining a cluster of 16 arrays of 4 elements each that are multiplied with
// the same matrix
#define NR_ARRAYS 32
#define NR_ARRAYS_ELEMS (NR_ARRAYS * ARR_SIZE)

// how many arr * mat multiplications can be performed on the ConnexArray in a
// round aka. in a chunk
#define ARR_IN_CHUNK (PROCESS_AT_ONCE * VECTOR_ARRAY_SIZE / MAT_SIZE_C)

#define MIN(x, y) ((x > y) ? y : x)


namespace gr {
  namespace opincaa {

    class multiply_cc_impl : public multiply_cc
    {
     private:
      ConnexMachine *connex;

      // Factors required for scaling the data
      int factor_mult1;
      int factor_mult2;
      int factor_res;

      void forecast(
        int noutput_items,
        gr_vector_int &ninput_items_required);

      /* \brief scales and casts a number of nr_elems input elements in an array
       *        of size ARR_SIZE
       */
      void prepareInArrConnex(
        uint16_t *out_arr, const gr_complex *in_arr, const int nr_elems);

      /* \brief scales and casts a number of nr_elems input elements in an
       *        matrix of size MAT_SIZE x MAT_SIZE
       */
      void prepareInMatConnex(
        uint16_t *out_mat, const gr_complex *in_mat, const int nr_elems);

      /* \brief Scales back the output data and converts the array of
       *        real and imaginary parts to a gr_complex array
       * \param out_data        pointer to a part of memory of at least nr_elems
       *                        elements
       * \param raw_out_data    data that will be coverted; pointer to a chunk
       *                        of memory of at least 2 * nr_elems elements
       * \param nr_elements     how many complex elements are "prepared"
       */
      void prepareOutDataConnex(
        gr_complex *out_data,
        const int32_t *raw_out_data,
        const int nr_elems);

      /* \brief multiplies the line arrays and the column arrays in a chunk
       */
      void multLineCol(
        std::vector<gr_complex> result,
        const gr_complex *line_arr,
        const gr_complex *col_arr,
        const int nr_chunk);

     public:
      multiply_cc_impl(
        std::string distributionFIFO,
        std::string reductionFIFO,
        std::string writeFIFO,
        std::string readFIFO);
      ~multiply_cc_impl();

      // Where all the action really happens
      int work(int noutput_items,
         gr_vector_const_void_star &input_items,
         gr_vector_void_star &output_items);
    };
  } // namespace opincaa
} // namespace gr

#endif /* INCLUDED_OPINCAA_MULTIPLY_CC_IMPL_H */
                \end{lstlisting}
