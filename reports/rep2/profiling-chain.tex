%=============================================================================
% SECTION 4 - The Profiling Chain
%=============================================================================

% TODO maybe change title...
\section{The Profiling Chain}
\label{sec:profiling-chain}
% TODO is it important - for the profiling - to change the AoA to see the
% performance? if not, how do i motivate this?
\subsection{Profile Methodology}

The GNU Radio Companion generates a Python script based on the flow graph
created in the graphic interface. The Python script then uses SWIG
\cite{cite:swig}, a tool that enables it to access the modules written in C++
and execute the flow graph.  Because of this intermediary step, the profiling
performed on the script is not very conclusive, since functions needed in the
wrapping process that are not directly related to the algorithm itself end up
consuming much of the execution time. For this reason, the preferred solution
was to create a C++ program that creates a flow graph similar to the initial
one, which also eliminates the dependency on a visual interface. \\

In this test program all the Qt blocks previously used are eliminated and, where
necessary, they are replaced with \textbf{Null Sink} blocks, which simply
discard the data feeding them, so that there are no time consuming I/O
operations. Additionally, the \textbf{Throttle} block is also discarded because
we want to profile the program when it runs at full speed. \\

After the program starts the flow graph, it creates a new thread and, after it
sleeps for 2 seconds (a period of time that can be adjusted), it calls for the
flow graph to stop its execution, otherwise it would run indefinitely. The full
code of the C++ flow graph and its Makefile can be found in Section
\ref{sec:code-snips}, in Listing \ref{lst:cpp-fg} and Listing
\ref{lst:cpp-fg-makefile}, respectively. \\

% TODO the hardware events vary with the processor - if we also profile it on
% ARM we should also be sure that we are comparing quite the same things
For profiling the processing chain, both \textbf{perf} \cite{cite:perf} and
\textbf{callgrind} \cite{cite:callgrind} have been used. \textbf{perf} is a
powerful tool for measuring a large set of hardware events, which also displays
assembly code for functions of interests, which can be helpful when wanting to
target micro-optimizations.  With \textbf{callgrind}, on the other hand, there
is the option of using the visual interface \textbf{kcachegrind}
\cite{cite:kcachegrind} for an easier tracing of the call graph. In this case,
the percentage of time spent in different functions is of interest, and both of
the profiling tools gave similar results. \\

Using \textbf{perf}, we can obtain how many cycles are spent in each function by using
the command:
\begin{lstlisting}[language=bash]
perf record -e cycles ./run_MUSIC_profile
\end{lstlisting}

If we want to view the call graph, we can instead use:
\begin{lstlisting}[language=bash]
perf record --call-graph lbr ./run_MUSIC_profile
\end{lstlisting}

Profiling the code with \textbf{callgrind} can be done with the following
command:
\begin{lstlisting}[language=bash]
valgrind --tool=callgrind ./run_MUSIC_profile
\end{lstlisting}

Since the results with both of the profilers are similar, only the ones obtained
with \textbf{perf} will be outlined. Of particular interest is the case when the
maximum level of optimization is being used, but since the call succession
cannot be observed very well in this case, another analysis with the lowest
level of optimization and debug symbols has been performed. The profiling flow
graph uses functions from the \textit{gr-doa} library of Ettus Research, as well
as from \textit{Armadillo} \cite{cite:armadillo}, a C++ library for linear
algebra, which was compiled from sources, as the latest version needed cannot be
found yet in the official repositories. The processing blocks rely internally
heavily on functions implemented in \textit{Armadillo} which may end up being
computing intensive, so when compiling the profiling chain with level 0 of
optimization, we have to make sure that both of the libraries are compiled in
the same way in order to see the hot spots of the processing chain.

\subsection{Profile results}

Using the maximum level of optimization, the most important profiling results
(for the functions which consume more than 2\% of the execution time) are
presented in Table \ref{tab:prof-o3}. \\

%=============================================================================
% Profiling table
%=============================================================================
\begin{table}[!htb]
\begin{center}
 \begin{tabular}{||c c c||} 
 \hline
 Overhead  & Command & Symbol \\ [0.5ex] 
 \hline\hline
 49,14\% 
 &
 find\_local\_max6
 &
 \makecell{arma::glue\_hist::apply\_noalias<unsigned int>}
 \\ 
 
 \hline
 13,73\%
 &
 autocorrelate4
 &
 \makecell{cgemm\_}
 \\
 
  \hline
 10,51\%
 &
 MUSIC\_lin\_array
 &
 \makecell{cgemm\_}
 \\
 
 \hline
 3,63\%
 &
 multiply\_matrix
 &
 \makecell{\_\_mulsc3}
 \\

  \hline
 3,56\%
 &
 MUSIC\_lin\_array
 &
 \makecell{cgemv\_}
 \\

  \hline
 3,06\%
 &
 MUSIC\_lin\_array
 &
 \makecell{gr::doa::MUSIC\_lin\_array\_impl::work}
 \\

  \hline
 2,33\%
 &
 MUSIC\_lin\_array
 &
 \makecell{\_\_logf\_finite}
 \\

  \hline
 2,08\%
 &
 multiply\_matrix  
 &
 \makecell{gr::blocks::multiply\_matrix\_cc\_impl::work}
 \\ [1ex] 
 \hline
\end{tabular}
\end{center}
\caption{Profiling Results With -O3}\label{tab:prof-o3}
\end{table}


According to the documentation of the tool, \textit{Overhead} is the percentage
of time spent in a particular function, \textit{Command} is the name of the task
that can be read via \code{/proc/<pid>/comm}, and \textit{Symbol} is the name of
the function executed at the moment of sampling. \\

Table \ref{tab:prof-o0} presents the output produced by \code{perf report} when
the libraries and the project are compiled with the lowest level of
optimization. \\

%=============================================================================
% Profiling table
%=============================================================================
\begin{table}[H]
\begin{center}
 \begin{tabular}{||c c c||} 
 \hline
 Overhead  & Command & Symbol \\ [0.5ex] 
 \hline\hline
 49,87\% 
 &
 find\_local\_max6
 &
 \makecell{arma::glue\_hist::apply\_noalias<unsigned int>}
 \\ 
 
 \hline
 6,36\%
 &
 autocorrelate4
 &
 \makecell{cgemm\_}
 \\
 
 \hline
 4,81\%  
 &
 autocorrelate4
 &
 \makecell{std::conj<float>}
 \\
 
 \hline
 4,66\%
 &
 MUSIC\_lin\_array
 &
 \makecell{std::complex<float>::complex}
 \\
 
 \hline
 3,38\%
 &
 MUSIC\_lin\_array
 &
 \makecell{cgemm\_}
 \\

 \hline
 2,46\%
 &
 autocorrelate
 &
 \makecell{arma::eop\_core<arma::eop\_conj>::apply \\
 <arma::Mat<std::complex<float> >, \\
 arma::Mat<std::complex<float> > >}
 \\

 \hline
 1,82\%
 &
 MUSIC\_lin\_array
 &
 \makecell{arma::Mat<std::complex<float> >:: \\
 init\_warm}
 \\

 \hline
 1,57\%
 &
 autocorrelate4
 &
 \makecell{arma::op\_strans::apply\_mat\_noalias \\
 <std::complex<float>,\\
 arma::Mat<std::complex<float> > >}
 \\

 \hline
 1,54\%
 &
 multiply\_matrix  
 &
 \_\_mulsc3
 \\

 \hline
 1,49\%
 &
 MUSIC\_lin\_array
 &
 cgemv\_
 \\ [1ex] 
 \hline
\end{tabular}
\end{center}
\caption{Profiling Results With -O0}\label{tab:prof-o0}
\end{table}


With further investigation, it is found that the most time consuming function is
\code{arma::glue\_hist::} \code{apply\_noalias<unsigned int>}, a function from
the \textit{Armadillo library} which is called from the function
\code{find\_local\_max}, responsible for finding the peaks in the MUSIC
spectrum. More precisely, the function is called from the \code{hist} functions
which is used in Listing \ref{lst:find-local-max}. \\

\begin{minipage}{\linewidth}
\lstset{
	language=C++,
	directivestyle={\color{black}},
	emph={int,char,double,float,unsigned},
	emphstyle={\color{RoyalBlue}}
    }		
\begin{lstlisting}[
	firstnumber=105,
	caption={Computationally Intensive Part in find\_local\_max},
	label={lst:find-local-max}
    ]
// finding all peak locations by computing
// second-order difference of
// sign of first-order difference
uvec indx2 = find(diff(sign_fod_in_vec) == -2)+1;

uvec indx3;
uvec indx3_unique;
uvec hist_indx3;
uvec all_pk_indxs;
if (!indx2.is_empty())
{
    // finding set-intersection between indx1 and indx2
    // in order to identify local peaks
    indx3 = join_vert(indx1, indx2);
    indx3_unique = unique(indx3);
    hist_indx3 = hist(indx3, indx3_unique);

    all_pk_indxs = indx1(find(hist_indx3 == 2));
}
\end{lstlisting}
\end{minipage}

TODO: explain what that part isn't necessary in most of the cases and which are
those \\

By eliminating that processing overhead where it is not necessary, we obtain the
new profiling results in Table \ref{tab:prof-remove}, where only the functions
that make up more than 5\% of the execution time are presented. 

%=============================================================================
% Profiling table
%=============================================================================
\begin{table}[H]
\begin{center}
 \begin{tabular}{||c c c||} 
 \hline
 Overhead  & Command & Symbol \\ [0.5ex] 
 \hline\hline
 12,73\% 
 &
 autocorrelate8
 &
 \makecell{cgemm\_}
 \\ 
 
 \hline
 11,50\%
 &
 MUSIC\_lin\_array
 &
 \makecell{cgemm\_}
 \\
 
 \hline
 7,93\%  
 &
 noise\_source\_c3
 &
 \makecell{\_\_ieee754\_log\_avx}
 \\
 
 \hline
 7,89\%  
 &
 noise\_source\_c4
 &
 \makecell{\_\_ieee754\_log\_avx}
 \\
 
 \hline
 6,98\%  
 &
 noise\_source\_c3
 &
 \makecell{gr::random::ran1}
 \\
 
 \hline
 6,42\%  
 &
 noise\_source\_c4
 &
 \makecell{gr::random::gasdev}
 \\
 
 \hline
 6,09\%  
 &
 noise\_source\_c4
 &
 \makecell{gr::random::ran1}
 \\
 
 \hline
 5,32\%  
 &
 noise\_source\_c3
 &
 \makecell{gr::random::gasdev}
 \\
 
 \hline
 5,32\%  
 &
 MUSIC\_lin\_array
 &
 \makecell{gr::doa::MUSIC\_lin\_array\_impl::work}
 
 \\ [1ex] 
 \hline
\end{tabular}
\end{center}
\caption{Profiling Results After Corrections}\label{tab:prof-remove}
\end{table}


Out of them, the ones related to the noise source are not of interest, since
they are relevant only for the simulation process, and we will therefore focus
only on the first two results. \\

Using the profiling tools, we find out that the hotspot in the \textbf{MUSIC
Linear Array} block is the computation of the MUSIC spectrum using
\eqref{eq:music-spectrum}, which can be found in Listing \ref{lst:hot-music}, on
line 136. \\

\begin{minipage}{\linewidth}
\lstset{
	language=C++,
	directivestyle={\color{black}},
	emph={int,char,double,float,unsigned},
	emphstyle={\color{RoyalBlue}}
    }		
\begin{lstlisting}[
	firstnumber=134,
	caption={Computationally Intensive Part in the MUSIC Linear Array block},
	label={lst:hot-music}
    ]
for (int ii = 0; ii < d_pspectrum_len; ii++)
{
  Q_temp = as_scalar(d_vii_matrix_trans.row(ii) * U_N_sq * d_vii_matrix.col(ii));
  out_vec(ii) = 1.0 / Q_temp.real();
}
\end{lstlisting}
\end{minipage}

As far as the autocorrelation is concerned, the computationally intensive part
is done in the first part of the calculation of the autocorrelation,
corresponding to formula \eqref{eq:autocorr-first-part}. 
Listing \ref{lst:hot-autocorrelation} depicts the part of the code
that performs this specific function. The \code{d\_input\_matrix} is already
stored in a transposed form, which explains the differences between the formula
and the code. \\

\begin{minipage}{\linewidth}
\lstset{
	language=C++,
	directivestyle={\color{black}},
	emph={int,char,double,float,unsigned},
	emphstyle={\color{RoyalBlue}}
    }		
\begin{lstlisting}[
	firstnumber=105,
	caption={Computationally Intensive Part in autocorrelate},
	label={lst:hot-autocorrelation}
    ]
out_matrix =
    (1.0 / d_snapshot_size) * d_input_matrix.st() * conj(d_input_matrix);
\end{lstlisting}
\end{minipage}

\subsubsection{Profile results on the Xilinx ZedBoard Zynq-7000 ARM/FPGA SoC
Development Board}

% TODO: how many details about the techincal specifications?
The ConnexArray SIMD processor that will be used for offloading certain parts of
the MUSIC algorithm is implemented on the FPGA on a Zedboard Zynq-7000
development board \cite{cite:zedboard}. Therefore, we are also interested in
profiling the application on the processor on the board, which is a Dual-Core
ARM Cortex A9 processor. The time ratio spent in the main functions is similar
to the previous results, the only difference lying in the number of samples
processed in the same amount of time (in this case, 30 seconds) during which the
chain is running - 5664 compared with 69879 on the Intel processor. \\

Table \ref{tab:prof-zedboard} presents the profiling results on the Dual-Core
ARM Cortex A9 processor, leaving only the results that end up consuming more
than 5\% of the processing time.

%=============================================================================
% Profiling table
%=============================================================================
\begin{table}[H]
\begin{center}
 \begin{tabular}{||c c c||} 
 \hline
 Overhead  & Command & Symbol \\ [0.5ex] 
 \hline\hline
 15,19\% 
 &
 autocorrelate8
 &
 \makecell{cgemm\_}
 \\ 
 
 \hline
 10,73\%
 &
 MUSIC\_lin\_array
 &
 \makecell{cgemm\_}
 \\
 
 \hline
 5,73\%  
 &
 noise\_source\_c4
 &
 \makecell{\_\_logf\_finite}
 \\
 
 \hline
 5,62\%  
 &
 noise\_source\_c3
 &
 \makecell{\_\_logf\_finite}

 \\ [1ex] 
 \hline
\end{tabular}
\end{center}
\caption{Profiling Results After Corrections on ARM Cortex A9}\label{tab:prof-zedboard}
\end{table}



