%=============================================================================
% SECTION 2 - Steps in MUSIC Algorithm
%=============================================================================

\section{Outline of the MUSIC Algorithm}
\label{sec:steps-music-algo}
The theory behind the MUSIC algorithm discussed in section
\S\ref{sec:theory-music} can be summarized in some elementary steps. We remind
that M is the number of elements in the antenna array and D is the number of
incoming signals at each element.

\subsection*{Step 1}
\label{ssec:step1}
Knowing the signal $x_i$ arriving at the element number \textit{i} of the
antenna array, the input covariance matrix can be computed:
\begin{equation}
    \bm{R}_{xx} = E[\bm{xx}^H]
\end{equation}
\begin{equation}
    x = \begin{bmatrix}x_1 & x_2 & ... & x_M \end{bmatrix} \\
\end{equation}

\subsection*{Step 2}
\label{ssec:step2}
Estimate the number of signals arriving at each element of the array. In order
to do this, we need to compute the eigenvalues of $R_{xx}$ and, from the
multiplicity $K$ of the smallest eigenvalue, we estimate the number of signals
as follows:
\begin{equation}
    \hat{D} = M - K
\end{equation}

\subsection*{Step 3}
\label{ssec:step3}
Compute the MUSIC spatial spectrum using
\begin{equation}
    P_{MUSIC}(\theta) =
        \frac{\bm{a}^H(\theta)\bm{a}(\theta)}
             {\bm{a}^H(\theta)\bm{V}_N\bm{V}_N^H\bm{a}(\theta)} \\
\end{equation}
\begin{equation}
    \bm{V}_N = \begin{bmatrix}v_{D+1}, & ..., & v_M \end{bmatrix}
\end{equation}

\subsection*{Step 4}
\label{ssec:step4}
Find the peaks of the estimated MUSIC spectrum which give us direction of
arrival for the $\hat{D}$ signals.
