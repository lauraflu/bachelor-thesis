%============================================================================
% CODE SNIPPETS
%============================================================================

\newpage
\section{Code snippets}
\label{sec:code-snips}

	\subsection{The C++ Flow Graph Used for Profiling}
	\label{ssec:code-profiling}
		\lstset{
		    language=C++,
		    directivestyle={\color{black}},
		    emph={int,char,double,float,unsigned},
		    emphstyle={\color{RoyalBlue}}
		}		
		\begin{lstlisting}[
		    caption={The C++ Flow Graph Used for Profiling},
		    label={lst:cpp-fg}
		]
#include <gnuradio/top_block.h>
#include <gnuradio/analog/sig_source_c.h>
#include <gnuradio/blocks/multiply_matrix_cc.h>
#include <gnuradio/blocks/null_sink.h>
#include <gnuradio/blocks/vector_to_streams.h>
#include <gnuradio/blocks/file_sink.h>
#include <gnuradio/blocks/vector_sink_f.h>
#include <gnuradio/blocks/vector_sink_c.h>
#include <gnuradio/gr_complex.h>

#include <doa/autocorrelate.h>
#include <doa/MUSIC_lin_array.h>
#include <doa/find_local_max.h>

#include <cmath>
#include <complex>
#include <chrono>
#include <thread>
#include <boost/math/constants/constants.hpp>

struct input_variables
{
    double sample_rate;
    double tone_freq1;
    double tone_freq2;
    float norm_spacing;
    int num_targets;
    int num_array_elements;
    int p_spectrum_length;
    int snapshot_size;
    int overlap_size;
};


int main(int argc, char **argv)
{
    using namespace std::complex_literals;

    gr::top_block_sptr fg = gr::make_top_block("run_MUSIC");

    const float pi = boost::math::constants::pi<float>();

    /*=============================================
     * Variables
     *============================================*/
    struct input_variables in_vars = {
        320000, // sample_rate
        10000,  // tone_freq1
        20000,  // tone_freq2
        0.4,    // norm_spacing
        2,      // num_targets
        4,      // num_array_elements
        1024,   // p_spectrum_length
        2048,   // snapshot_size
        512     // overlap_size
    };
    int theta0_deg = 50;
    int theta1_deg = 140;
    float theta0 = pi * theta0_deg / 180;
    float theta1 = pi * theta1_deg / 180;
    
    std::vector<gr_complex> amv0(in_vars.num_array_elements);
    std::vector<gr_complex> amv1(in_vars.num_array_elements);
    std::vector<std::vector<gr_complex>> array_manifold_matrix(
        in_vars.num_array_elements,
        std::vector<gr_complex>(in_vars.num_targets)
    );
    float ant_loc_val;
    
    if (in_vars.num_array_elements % 2 == 1) {
        ant_loc_val = (float)in_vars.num_array_elements / 2;
    } else {
        ant_loc_val = (float)in_vars.num_array_elements/2 - 0.5;
    }

    // can't multiply by an integer (2) when we have -1if
    float temp_norm_spacing = in_vars.norm_spacing * 2;
    for (int i = 0; i < in_vars.num_array_elements; i++) {
        amv0[i] = std::exp(-1if * temp_norm_spacing * pi * 
			    ant_loc_val * cosf(theta0));
        amv1[i] = std::exp(-1if * temp_norm_spacing * pi *
			    ant_loc_val * cosf(theta1));
        array_manifold_matrix[i][0] = amv0[i];
        array_manifold_matrix[i][1] = amv1[i];

        ant_loc_val--;
    }
    
    /*=============================================
     * Blocks
     *============================================*/
    
    gr::analog::sig_source_c::sptr sig_src1 = 
        gr::analog::sig_source_c::make(
            in_vars.sample_rate,
            gr::analog::GR_COS_WAVE, // waveform used
            in_vars.tone_freq1,     // wave frequency
            1,                      // amplitude
            0                       // offset
        );
    
    gr::analog::sig_source_c::sptr sig_src2 = 
        gr::analog::sig_source_c::make(
            in_vars.sample_rate,
            gr::analog::GR_COS_WAVE, // waveform used
            in_vars.tone_freq2,     // wave frequency
            1,                      // amplitude
            0                       // offset
        );

    gr::blocks::multiply_matrix_cc::sptr mult_by_matrix1 = 
        gr::blocks::multiply_matrix_cc::make(
            array_manifold_matrix,
            gr::block::TPP_ALL_TO_ALL
        );

    gr::doa::autocorrelate::sptr autocorr1 = 
        gr::doa::autocorrelate::make(
            in_vars.num_array_elements,
            in_vars.snapshot_size,
            in_vars.overlap_size,
            1 
        );
    
    gr::doa::MUSIC_lin_array::sptr music_lin_array1 =
        gr::doa::MUSIC_lin_array::make(
            in_vars.norm_spacing,
            in_vars.num_targets,
            in_vars.num_array_elements,
            in_vars.p_spectrum_length
        );

    gr::doa::find_local_max::sptr find_local_max1 = 
        gr::doa::find_local_max::make(
            in_vars.num_targets,
            in_vars.p_spectrum_length,
            0.0, // min value of index vector
            180.0 // max value of index vector
        );

    // Null sink for locations
    gr::blocks::null_sink::sptr null_sink1 = 
        gr::blocks::null_sink::make(
            sizeof(float) * in_vars.num_targets // item size
        );

#ifdef PROFILE_GR_DOA    
    // Null sink for directions of arrival ---> for profiling
    gr::blocks::null_sink::sptr null_sink2 = 
        gr::blocks::null_sink::make(
            sizeof(float) * in_vars.num_targets // item size
        );
#else
    // Vector to streams + Vector sinks ---> for testing
    gr::blocks::vector_to_streams::sptr v_to_s1 = 
        gr::blocks::vector_to_streams::make(
            sizeof(float),      // item size
            in_vars.num_targets // number of streams
        );
    gr::blocks::vector_sink_f::sptr v_sink1 = 
        gr::blocks::vector_sink_f::make();
    gr::blocks::vector_sink_f::sptr v_sink2 = 
        gr::blocks::vector_sink_f::make();
#endif
   
    /*=============================================
     * Connections
     *============================================*/

    fg->connect(sig_src1, 0, mult_by_matrix1, 0);
    fg->connect(sig_src2, 0, mult_by_matrix1, 1);
    fg->connect(mult_by_matrix1, 0, autocorr1, 0);
    fg->connect(mult_by_matrix1, 1, autocorr1, 1);
    fg->connect(mult_by_matrix1, 2, autocorr1, 2);
    fg->connect(mult_by_matrix1, 3, autocorr1, 3);
    fg->connect(autocorr1, 0, music_lin_array1, 0);
    fg->connect(music_lin_array1, 0, find_local_max1, 0);
    fg->connect(find_local_max1, 0, null_sink1, 0);
   
    
#ifdef PROFILE_GR_DOA    
    // Null sink ---> for profiling
    fg->connect(find_local_max1, 1, null_sink2, 0);
#else     
    // Vector to streams + Vector sinks ---> for testing
    fg->connect(find_local_max1, 1, v_to_s1, 0);
    fg->connect(v_to_s1, 0, v_sink1, 0);
    fg->connect(v_to_s1, 1, v_sink2, 0);
#endif

    /*=============================================
     * Run the graph
     *============================================*/
    fg->start();

    std::thread t([&fg]{
        // std::cout << "Sleeping...\n";
        std::this_thread::sleep_for(std::chrono::seconds(2));
        fg->stop();
        fg->wait();
    });
    t.join();

#ifndef PROFILE_GR_DOA    
    // For testing
    std::vector<float> out_data1 = v_sink1->data();
    std::vector<float> out_data2 = v_sink2->data();
    std::cout << "angle1 = " << out_data1[0] << std::endl;
    std::cout << "angle2 = " << out_data2[0] << std::endl;
#endif

    return 0;
}
		\end{lstlisting}
	
	\subsection{Makefile for the C++ Tester}
	\label{ssec:makefile-profiling}
		\begin{lstlisting}[
		    language=make,
		    caption={Makefile for the C++ Flow Graph},
		    label={lst:cpp-fg-makefile}
		]
WD = $(shell pwd)
GR_DOA_FP = /home/work/gr-doa
CXXFLAGS = -I$(GR_DOA_FP)/include -O0 -std=c++14 -g
LDFLAGS = -L$(GR_DOA_FP)/build/lib/
LDLIBS = -lgnuradio-runtime -lboost_system -lgnuradio-blocks \
	 -lgnuradio-pmt -lgnuradio-analog -lgnuradio-doa \
	 -lpthread

all: run_MUSIC
profile: run_MUSIC_profile

run_MUSIC: run_MUSIC.cc
	g++ $< $(LDFLAGS) $(CXXFLAGS) $(LDLIBS) -o $@

run_MUSIC_profile: run_MUSIC.cc
	g++ $< $(LDFLAGS) $(CXXFLAGS) -DPROFILE_GR_DOA $(LDLIBS) -o $@

clean:
	rm run_MUSIC run_MUSIC_profile
	\end{lstlisting}


